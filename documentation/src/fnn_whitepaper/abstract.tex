%!TEX root = main.tex
\textit{\textbf{NOTE: This is an old abstract for only the first part of the paper (FNNs).}}In this paper we consider the traditional model of feed-forward neural networks proposed in (McCulloch and Pitts, 1949), and using intuitions developed in (Neal, 1994) we propose a method generalizing discrete neural networks as follows. In the standardized case, neural mappings $\mathcal{N}: \mathbb{R}^n \to [0,1]^m$ have little meaning when $n \to \infty$. Thus we consider a new construction $\mathcal{F}: \mathcal{X} \to \mathcal{Y}$ where the domain and codomain of $\mathcal{N}$ become infinite dimensional Hilbert spaces, namely the set of quadratically Lebesgue integrable functions $L^2$ over a real interval $E$ and $[0,1]$ respectively. The derivation of this construction is intuitively similar to that of Lebesgue integration; that is, $\sum_i \sigma_i w_{ij} \to \int_{E\subset\mathbb{R}}\sigma(i)w(i,j)\ d\mu(s)$. 

After establishing a proper family of "functional neural networks" $\mathcal{F}$, we show that $\mathcal{N}$ are a specific class of functional neural networks under specific constraints. More specifically in our first lemma, we prove that $\mathcal{F} \equiv \mathcal{N}$ for piecewise constant weight functions $w(i,j)$. Having done so, we then attempt to find an analogue to Cybenko's theorem of universal approximation for neural networks. Firstly, we prove as a corollary of the Weierstrass approximation theorem, that $w(i,j)$ can approximate a function,$f:E\to[0,1]$, satisfying $\left\|\mathcal{F}\xi - f(\xi)\right\|_\infty \to 0$. As a byproduct of the proof, we also establish a closed-form definition for the satisfying $w(i,j)$ and thereby through our first lemma provide novel insight into the actual form of the weight matrix $[w_{ij}]$ for trained $\mathcal{N}$. Finally we propose a universal approximation theorem for functional neural networks; that is, we show through the Riesz Representation Theorem that $\mathcal{F}$ approximates any bounded linear operator on $\mathcal{X}$.

In conclusion, we create a practical analogue of the error-backpropagation algorithm, and implement functional neural networks using Simpson’s rule. We suggest that functional neural networks represent an interesting opportunity for the implementation of machine learning systems modeling functional transformation.